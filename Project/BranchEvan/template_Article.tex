\documentclass[]{article}

%opening
\title{Shallow Water Equations - Optimization Strategy}
\author{Gaurav Kukreja, Evangelos Drossos}

\begin{document}

\maketitle

\begin{abstract}
Shallow Water Equations (SWE) are typically used for tsunami or dam-break simulations.
The numerical discretization techniques and resulting computational challenges can be
applied to other hyperbolic PDE's as well. Examples are Earthquake simulations,
weather and climate simulations or aerodynamics. The task of the project is to optimize
SWE code. 

Our objective for the project is to optimize the code by using Compiler Intrinsics to
improve performance from vectorization. Further, we will parallelize the computation using
OpenMP and MPI. We will also try to improve Cache Performance by using Cache Blocking
and compare results by using different strategies for the same. A detailed description of
our approach is discussed in this report.
\end{abstract}

\section{Compiler Intrinsics to Optimize Solvers}
The code under \textit{src/solvers} contains the implementation of the stencil operations.
This code can be optimized by using Compiler Intrinsics for improving performance from vectorization.
Multiple approaches for solvers are implemented. We will focus on optimizing the 
F-wave solver. All functions are inline. The Intel Compiler is capable of vectorizing
the code automatically. However, the performance can be further improved by using Compiler
Intrinsics. We plan to concentrate on optimizing our code for single precision floating point values.

If time permits, we will try to optimize the Augmented Riemann Solvers as well.

\section{Cache Blocking}
We will implement various Cache Blocking Optimizations using existing SWE\_Blocks and
implementing blocking within each SWE\_Block. We will to compare our implementations and
produce a report based on our understanding of the bottleneck issues.

If time permits, we will try to extend the analysis using Scalasca to better identify
the bottlenecks.

\section{Parallelization using OpenMP and MPI}
We will try to find the best strategy to parallelize our implementation using hybrid
approach with OpenMP and MPI. 

We will partition the problem space into 2 dimensional blocks using \textit{SWE\_block}. Computation
for each block will be performed in parallel using MPI. Each block will be executed on a node.
For this, we will need to exchange values of the border edges for each iteration, which will
be done using MPI\_Vector and MPI constructs for sending and receiving messages. We will try
to optimize this by overlapping communication with computation.
By creating a pipeline we will try to maximize the performance that can be achieved.

The computation inside SWE\_block can be parallelized
using OpenMP, and will be executed on a single node. This is so that message passing
over the network can be reduced.

\section{Performance Analysis using Scalasca}
If time permits, we will try to manually instrument the SWE code. We will present our
understanding about the bottleneck issues and demonstrate how our implementation
resolves these issues.

\section{Distribution of Tasks}
We will start our implementation with tasks of vectorization and hybrid parallelization
with OpenMP and MPI. Gaurav will focus on parallelization while Evangelos will work
on improving vectorization using Compiler Intrinsics. 

We will then collaborate on identifying Cache bottlenecks and solve them by implementing
Cache Blocking strategies. We hope to be able to complete these tasks by 18 January 2014.

We will work on fixing bugs if any, and producing results to demonstrate the merits of our
implementation and identify future scope. We will try to instrument the code using Scalasca
and extend our analysis to support our understanding of the problem and demonstrate how our implementation mitigates the problems.

\end{document}